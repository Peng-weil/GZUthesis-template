\documentclass[]{GZUthesis}

\begin{document}
\maketitle
\begin{abstract}
    挖掘隐藏在异质信息网络中丰富的语义信息是数据挖掘的重要任务之一.离群点在值、数据分布、和产生机制上都明显不同于正常数据对象.检测离群点并分析其不同的产生机制,最终消除离群点具有重要的现实意义.目前,针对异质信息网络动态离群点检测的研究工作相对较少,还有很多问题有待解决.由于异质信息网络的动态性,随着时间的变化,正常数据对象也可能转变为离群点.针对异质网络提出一种基于张量表示的动态离群点检测方法,并根据张量表示的高阶数据构建张量索引树.通过搜索张量索引树,将特征加入到直接项集和间接项集中.同时,根据基于短文本相关性的聚类方法来判断数据集中的数据对象是否偏离其原聚簇来动态检测网络中的离群点.该模型能够在充分降低时间和空间复杂度的条件下保留异质网络中的语义信息.实验结果表明,该方法能够快速有效地进行异质网络环境下的动态离群点检测
\end{abstract}
\newline
\keywords{网络}
\newline
\begin{abstract}[jj]
    sss
\end{abstract}
ssssdsddddd

love \underline{Emphasized} text and \underline{another} and \underline{other}

\end{document}
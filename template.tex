\documentclass[]{GZUthesis}
\graduationyear{2021}
\major{学科专业}
\research{研究方向}
\author{学生}
\supervisor{导师}
\stuid{(学号)}
\title{论文题目}

\chairman{贵州大学\ 张XX\ 教授}
\expert{贵州大学\ 李XX\ 教授}
\expert{贵州大学\ 李XX\ 教授}
\expert{贵州大学\ 李XX\ 教授}
\expert{贵州大学\ 刘XX\ 教授}
\secretary{张三\ 学士}
\addbibresource{reference.bib} % 参考文献

\begin{document}

\maketitle
\frontmatter
\begin{abstract}
	这是一段摘要。
\end{abstract}
\keywords{计算机;人工智能}
\begin{abstract}[abstract]
	this is abstract.
\end{abstract}
\keywords[Key words:]{PC; AI}
\tableofcontents
\mainmatter
\chapter{引言}
\section{二级目录}
\subsection{三级目录}
20世纪90年代,物联网的出现早已出现端倪。比尔盖茨曾在他的著作《未来之路》中,描写了未来世界物联网出现的场景,但是其并未对这种未来技术命名。\cite{gates1995road}同时期,就职于美国麻省理工学院的Kevin Ash-ton教授首次提出了物联网的概念。90年代末期,美国麻省理工学院提出“万物皆可通过网络互联”的思想,并依托于射频识别(RFID)技术得以实现。
中文引用\cite{孙其博2010物联网}

% 致谢
\begin{thank}
	听我说谢谢你因为有你温暖了四季。
\end{thank}
\printbibliography

\begin{appendices}
	\chapter{参与会议}
\end{appendices}
\promise
\end{document}
\documentclass[version=evaluate,type=doctor,oneside]{GZUthesis} % 盲评博士论文,单面打印
\graduationyear{2021}
\major{学科专业}
\research{研究方向}
\author{学生}
\supervisor{导师}
\stuid{(学号)}
\title{贵州大学研究生毕业论文模板v1.0.0}
\security{<密级>}
\classifyno{<分类号>}

\chairman{贵州大学\ 张XX\ 教授}
\expert{贵州大学\ 李XX\ 教授}
\expert{贵州大学\ 李XX\ 教授}
\expert{贵州大学\ 李XX\ 教授}
\expert{贵州大学\ 刘XX\ 教授}
\secretary{张三\ 学士}
\addbibresource{reference.bib} % 参考文献

\begin{document}

\maketitle
\frontmatter
\begin{abstract}
	这是一段摘要。
\end{abstract}
\keywords{计算机;人工智能}
\begin{abstract}[abstract]
	this is abstract.
\end{abstract}
\keywords[Key words:]{PC; AI}
\tableofcontents
\mainmatter
\chapter{模板说明}
\section{动机}
\section{介绍}
本模板提供贵州大学硕士博士以及本科生\footnote{本科生部分目前还不完善}版本的学位(毕业)论文模板。
本模板依据2022年发布《贵州大学研究生学位论文格式》、《贵州大学研究生学位论文盲评封面格式》制作;
本科生部分依据作者记忆制作\footnote{由于作者毕业太久又没有抓到本科生只好胡乱制作}。

同时本本模板提供三种输出模式分别是草稿、盲审、打印。两类布局模式:单页打印、双页打印。具体内容将在以后章节介绍。
\subsection{控制命令}

\begin{itemize}
    \item $\backslash$frontmatter 前言格式,页数格式为罗马底部居中;
\item $\backslash$mainmatter 正文格式,页眉为章节名称以及论文题目,页脚为数字页数居中;
\item $\backslash$chairman 答辩主席;
\item $\backslash$expert 答辩专家;
\item $\backslash$secretary 答辩秘书;
\item $\backslash$security 密级;
\item $\backslash$classifyno 分类号;
\item $\backslash$promise 打印承诺书 研究生默认添加到文章最后,本科生默认添加到封面之后。
\end{itemize}

\subsection{使用}
本文档在Tex live 2021 版本以及 MikTeX 22.3版本编译通过;\textbf{没有}在Linux/Mac 下测试。
基于 book 制作,使用 xelatex 编译。

\subsection{模板参数}
\begin{itemize}
    \item \textbf{type} 模板类型,doctor 为博士master 为硕士 bachelor为本科;
    \item \textbf{version} 打印类型 evaluate(盲评版本),print(打印版本),draft(草稿);
    \item 可以通过 \textbf{oneside}、\textbf{twoside} 两个参数控制单面打印或者双面打印,默认单面。双面时会留出装订边距。
\end{itemize}

当前打印格式为 盲评博士论文单面打印,详见第一行

\subsection{注意事项}
日期需使用中文分割如
\begin{verbatim}
    \date{20XX年XX月}
\end{verbatim}
 或者
 \begin{verbatim}
    \date{20XX年XX月XX日}
\end{verbatim} \par

\subsection{参考文献引用示例}
English\cite{gates1995road}
中文\cite{孙其博2010物联网}。
% 致谢
\begin{thank}
	听我说谢谢你因为有你温暖了四季。
\end{thank}
\printbibliography

\begin{appendices}
	\chapter{参与会议}
\end{appendices}
\end{document}

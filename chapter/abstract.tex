\begin{abstract}
    本研究探讨了深度神经网络在图像识别领域的最新应用和优化技术。通过对卷积神经网络(CNN)、循环神经网络(RNN)和生成对抗网络(GAN)的深入分析,提出了一种新的混合模型架构,显著提高了图像识别的准确率和效率。本文还探讨了神经网络在医疗图像分析、自动驾驶和安防系统等实际应用中的潜力和挑战。
\end{abstract}
\keywords{深度学习; 神经网络;图像识别; 卷积神经网络; 混合模型}


\begin{abstract}[abstract]
    This research investigates the latest applications and optimization techniques of deep neural networks in the field of image recognition. Through an in-depth analysis of Convolutional Neural Networks (CNNs), Recurrent Neural Networks (RNNs), and Generative Adversarial Networks (GANs), we propose a novel hybrid model architecture that significantly improves the accuracy and efficiency of image recognition tasks.
    
    Our study begins with a comprehensive review of traditional image recognition methods and the evolution of deep learning techniques. We then focus on the design and implementation of our proposed hybrid model, which combines the strengths of CNNs, RNNs, and attention mechanisms. Extensive experiments were conducted on benchmark datasets, demonstrating that our model outperforms existing state-of-the-art methods in terms of accuracy, computational efficiency, and generalization capability.
    
    Furthermore, this thesis explores the potential and challenges of applying neural networks in real-world scenarios, including medical image analysis, autonomous driving, and security systems. We provide case studies that illustrate the practical implications of our research and discuss the ethical considerations associated with deploying these technologies.
    
    Finally, we conclude by summarizing our main contributions, acknowledging the limitations of our approach, and proposing directions for future research. This work not only advances the theoretical understanding of deep neural networks but also provides practical insights for their application in various domains of image recognition.
    
\end{abstract}
\keywords[Key words:]{Deep Learning; Neural Networks; Image Recognition; Convolutional Neural Networks; Hybrid Models; Transfer Learning; Attention Mechanisms; Computer Vision}
\documentclass{ctxdoc}
\usepackage{ulem}
%\usepackage[markup=underlined,authormarkup=none,commentmarkup=todo]{changes}
\begin{document}
    \title{The \textsf{GZUthesis} package}
    \author{https://github.com/francisol/GZUthesis-template/}
    \maketitle

    \section*{更新历史}

    \begin{itemize}
        \item[1.0.0] (2022-08-17) 初始版本
        \begin{itemize}
            \item 间距格式调整
            \item 内置华文行楷
            \item 输出版本控制(完成部分)
            \item 兼容Mac/Linux
            \item 兼容 overleaf
            \item 答辩委员会名单
            \item 附录
            \item 摘要关键字
            \item 目录
        \end{itemize}

        \item[1.1.0] (2024-08-18) 重构
        \begin{itemize}
            \item 规避版权风险,删除了内置字体
            \item 增加代码注释
            \item 参考文献设置移除宏包
        \end{itemize}
    \end{itemize}


    \section{说明}
    \textbf{该文档为用户手册,并非宏包使用实例。}


    本模板提供贵州大学硕士博士\sout{以及本科生版本的}学位(毕业)论文模板。
    本模板依据2022年发布《贵州大学研究生学位论文格式》、《贵州大学研究生学位论文盲评封面格式》制作。

    本模板亦在\href{https://www.overleaf.com/latex/templates/gui-zhou-da-xue-yan-jiu-sheng-bi-ye-lun-wen-mo-ban/tstnysppbxvh}{Overleaf} 提供

    如有疑问请在 \href{https://github.com/francisol/GZUthesis-template/issues}{Github} 上提交 issues。

    \section{内置信息}
    \subsection{内置宏包}
    \begin{description}[align=left,leftmargin=!,labelwidth=5em]
        \item[xkeyval] 解析宏包参数,用于辅助编写宏包
        \item[ifthen] 用于辅助编写宏包
        \item[ctex] 中文字符处理
        \item[geometry] 设置页边距
        \item[hyperref] 链接
        \item[appendix] 实现附录
        \item[setspace] 设置行间距
        \item[titletoc] 设置目录
        \item[fancyhdr] 页面布局,实现页眉页脚
        \item[pdfpages] 插入pdf文件,实现插入承诺书
        \item[ulem] 用于实现下划线样式
        \item[bicaption] 用于实现中英图表说明
    \end{description}

    \subsection{内置样式}
    已增加中英文图表说明,图表默认五号黑体。

    参考文献样式未在宏包中限制,但在样例文件中给出国标格式。

    \section{已知问题}
    由于平台字体限制,除windows 平台外,其他平台不能按照模板规范显示,例如 linux系统使用的不是Sim*字体,部分扩展区文字显示乱码。 在没有 Time New Roman 和Arial字体时,模板会将字体分别指定为当前系统的默认字体和无衬线字体。

    中文字体由 \textbf{ctex}提供,模板并未修改。

    盲审版首页 \textbf{XX 届X士研究生学位论文} 字样为华文行楷(STXINGKA),若系统无该字体则使用楷体。


    \textbf{页面布局并未一比一复刻,但是没那么离谱。}


    \section{使用}
    \subsection{宏包选项}
    \begin{function}[added=2024-08-18]{degree}
        \begin{syntax}
            degree = <doctor|master|bachelor>
        \end{syntax}
        指定该文档的学位类型,默认为博士论文(doctor)。该值仅影响文档的标题。
    \end{function}
    \begin{function}[added=2024-08-18]{print}
        \begin{syntax}
            print = <review|final|draft>
        \end{syntax}
        指定该文档的打印类型,默认为草稿(draft)。该值影响文档的封面以及正文之前的页面展示。
    \end{function}

    \begin{optdesc}
        \item[review] 盲审版本,封面不显示个人信息,封面之后无附加信息。
        \item[final] 最终版本,封面显示个人信息,封面之后显示答辩信息,最后一页显示著作权信息。
        \item[draft] 草稿版本,用于自己校验,只显示正文。
    \end{optdesc}

    \subsection{宏包命令}

    本节介绍了本包自定义和重定义的命令。

    \subsubsection{信息设置类命令}
    该类命令均保存为\cs{GZU@*}变量

    \begin{function}[added=2024-08-18]{\title}
        \begin{syntax}
            \cs{title}\marg{论文题目} \\
        \end{syntax}
        该命令设置论文题目,题目展示最多三行,题目会写入到pdf文件属性中
        \begin{texnote}
            该命令为重定义命令
        \end{texnote}
    \end{function}

    \begin{function}[added=2024-08-18]{\author}
        \begin{syntax}
            \cs{author}\marg{作者姓名} \\
        \end{syntax}
        该命令设置论文作者,会填充到“研究生”、“答辩人”等表格内容中,作者会写入到pdf文件属性中
        \begin{texnote}
            该命令为重定义命令,不会覆盖\cs{\@author}。
        \end{texnote}

    \end{function}

    \begin{function}[added=2024-08-18]{\date}
        \begin{syntax}
            \cs{date}\marg{XXXX年XX月} \\
        \end{syntax}
        该命令设置日期,默认为当前时间,设置时间只会影响到封面中最后一行的年月信息显示。
        日期格式必须为 “XXXX年XX月” 或“XXXX年XX月xx日”,否则可能会报错或显示错误。
        \begin{texnote}
            该命令为重定义命令,不会覆盖\cs{\@date}。答辩时间不由此设置。
        \end{texnote}

    \end{function}


    \begin{function}[added=2024-08-18]{\paperno}
        \begin{syntax}
            \cs{paperno}\marg{论文号} \\
        \end{syntax}
        该命令设置论文编号,默认留空。
    \end{function}


    \begin{function}[added=2024-08-18]{\defsec}
        \begin{syntax}
            \cs{defsec}\marg{XXX+职称/学位} \\
        \end{syntax}
        该命令设置答辩秘书,格式为“XXX+职称/学位”,默认留空。
    \end{function}


    \begin{function}[added=2024-08-18]{\addexpert}
        \begin{syntax}
            \cs{addexpert}\marg{XXXXXX 大学 XXX+职称} \\
        \end{syntax}
        该命令为答辩专家列表添加一个专家信息,可以重复调用,只会追加信息。添加格式为 “XXXXXX 大学 XXX+职称”。
    \end{function}


    \begin{function}[added=2024-08-18]{\chairman}
        \begin{syntax}
            \cs{chairman}\marg{XXXXXX 大学 XXX+职称} \
        \end{syntax}
        该命令设置答辩主席,默认留空。格式为 “XXXXXX 大学 XXX+职称”。
    \end{function}


    \begin{function}[added=2024-08-18]{\major}
        \begin{syntax}
            \cs{major}\marg{学科专业} \\
        \end{syntax}
        该命令设置学科专业,默认留空。
    \end{function}


    \begin{function}[added=2024-08-18]{\class}
        \begin{syntax}
            \cs{class}\marg{班级} \\
        \end{syntax}
        该命令设置班级,默认留空。
        \begin{texnote}
            本科模板预留参数,研究生无效。
        \end{texnote}
    \end{function}


    \begin{function}[added=2024-08-18]{\college}
        \begin{syntax}
            \cs{college}\marg{学院} \\
        \end{syntax}
        该命令设置学院,默认留空。

        \begin{texnote}
            本科模板预留参数,研究生无效。
        \end{texnote}
    \end{function}


    \begin{function}[added=2024-08-18]{\research}
        \begin{syntax}
            \cs{research}\marg{研究方向} \\
        \end{syntax}
        该命令设置研究方向,默认留空。
    \end{function}


    \begin{function}[added=2024-08-18]{\supervisor}
        \begin{syntax}
            \cs{supervisor}\marg{导师} \\
        \end{syntax}
        该命令设置导师,默认留空。
    \end{function}


    \begin{function}[added=2024-08-18]{\stuid}
        \begin{syntax}
            \cs{stuid}\marg{学号} \
        \end{syntax}
        该命令设置学号,默认留空。

        \begin{texnote}
            本科模板预留参数,研究生无效。
        \end{texnote}
    \end{function}



    \begin{function}[added=2024-08-18]{\defaddr}
        \begin{syntax}
            \cs{defaddr}\marg{答辩地点} \
        \end{syntax}
        该命令设置答辩地点,默认留空。
    \end{function}


    \begin{function}[added=2024-08-18]{\gradyear}
        \begin{syntax}
            \cs{gradyear}\marg{毕业年份} \
        \end{syntax}
        该命令设置毕业年份,默认为当前年份。
    \end{function}


    \begin{function}[added=2024-08-18]{\defdate}
        \begin{syntax}
            \cs{defdate}\marg{答辩时间} \
        \end{syntax}
        该命令设置答辩时间,默认留空,免得忘记填写造成错误。
    \end{function}





    \begin{function}[added=2024-08-18]{\classifyno}
        \begin{syntax}
            \cs{classifyno}\marg{分类号}
        \end{syntax}
        该命令设置分中图类号,默认留空。
    \end{function}

    \subsubsection{控制类命令}

    \begin{function}[added=2024-08-18]{\maketitle}
        \begin{syntax}
            \cs{maketitle}
        \end{syntax}
        根据本文前述信息生成封面以及答辩信息(如果有)。
    \end{function}

    \begin{function}[added=2024-08-18]{\frontmatter}
        \begin{syntax}
            \cs{frontmatter}
        \end{syntax}
        正文前样式,无页眉,页脚只保留罗马数字的页码信息
    \end{function}

    \begin{function}[added=2024-08-18]{\mainmatter}
        \begin{syntax}
            \cs{mainmatter}
        \end{syntax}
        正文样式,页眉含有章节信息以及论文题目,页脚只保留阿拉伯数字的页码信息。
    \end{function}


    \begin{function}[added=2024-08-18]{\keywords}
        \begin{syntax}
            \cs{keywords}\\
            \cs{keywords}\marg{Key words:}
        \end{syntax}
        按要求格式输出关键字信息,接收一个可选参数作为关键字的名字,默认为 \textbf{关键字:}
    \end{function}



    \begin{function}[added=2024-08-18]{\acknowledgments}
        \begin{syntax}
            \cs{acknowledgments}\\
            \cs{acknowledgments}\marg{致谢}
        \end{syntax}
        生成致谢章节,章节不编码,标题居中。命令参数为章节标题,默认为 \textbf{致谢}
    \end{function}

    \subsection{宏包环境}

    \begin{environment}{abstract}
        \begin{syntax}
            \cs{begin}\{abstract\}[abstract] \\
            This research investigates the latest applications and optimization techniques
            of deep neural networks in the field of image recognition. \\
            \cs{end}\{abstract\}
        \end{syntax}
        这个环境用于创建摘要章节,章节名称通过可选参数指定,默认为 \textbf{摘要}。
        \begin{texnote}
            abstract环境在book中不存在。
        \end{texnote}
    \end{environment}


    \begin{environment}{appendices}
        \begin{syntax}
            \cs{begin}\{appendices\} \\
            \cs{chapter}\{参与会议\} \\
            \cs{end}\{appendices\}
        \end{syntax}
        这个环境用于创建附录章节,章节不编码,包裹在其中的章节按大写英文字母编码。

        \begin{texnote}
            使用示例:
            \begin{verbatim}
                \begin{appendices}
                    \chapter{参与会议}
                \end{appendices}
            \end{verbatim}

            目录会新增两个一级目录分别为 \textbf{附录} 、\textbf{附录 A 参与会议}。

        \end{texnote}

    \end{environment}

\end{document}